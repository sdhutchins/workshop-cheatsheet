\documentclass[9pt]{cheatsheet}

\cheatsheettitle{Git, Bash, and Programming Basics}

\begin{document}

\begin{multicols*}{3}

\cheatsheetsection{Git Basics}
\textbf{Git} is a distributed version control system commonly used for tracking changes in code.

\begin{enumerate}
  \item Initialize a Git repository: \texttt{git init}
  \item Clone a repository: \texttt{git clone <repository\_url>}
  \item Check repository status: \texttt{git status}
  \item Stage changes: \texttt{git add <file>}
  \item Commit changes: \texttt{git commit -m "Commit message"}
  \item Push to a remote repository: \texttt{git push}
  \item Pull changes from a remote: \texttt{git pull}
  \item Create a new branch: \texttt{git branch <branch\_name>}
  \item Switch to a branch: \texttt{git checkout <branch\_name>}
  \item Merge branches: \texttt{git merge <branch\_name>}
  \item View commit history: \texttt{git log}
  \item Discard local changes: \texttt{git reset --hard}
  \item Create a Git tag: \texttt{git tag <tag\_name>}
  \item Remove a Git tag: \texttt{git tag -d <tag\_name>}
\end{enumerate}

\cheatsheetsection{Bash/Command Line Basics}
\textbf{Bash (Bourne Again Shell)} is a command-line shell and scripting language.

\begin{enumerate}
  \item Navigate to a directory: \texttt{cd <directory\_path>}
  \item List files and directories: \texttt{ls}
  \item Create a new directory: \texttt{mkdir <directory\_name>}
  \item Remove a file: \texttt{rm <file\_name>}
  \item Copy a file: \texttt{cp <source\_file> <destination\_directory>}
  \item Move/Rename a file: \texttt{mv <old\_name> <new\_name>}
  \item Display file content: \texttt{cat <file>}
  \item Execute a script: \texttt{./<script\_name>}
  \item View manual page for a command: \texttt{man <command>}
  \item Find files by name: \texttt{find <directory> -name <filename>}
  \item Search for text in files: \texttt{grep <pattern> <file>}
  \item Redirect output to a file: \texttt{command > output.txt}
\end{enumerate}

\cheatsheetsection{Common Programming Words and Symbols}

\begin{itemize}
  \item \textbf{Variable}: A symbolic name for a value.
  \item \textbf{Function}: A reusable block of code.
  \item \textbf{Conditional Statement}: Executes code based on a condition (e.g., \texttt{if}, \texttt{else}).
  \item \textbf{Loop}: Repeats a block of code until a condition is met (e.g., \texttt{for}, \texttt{while}).
  \item \textbf{Comment}: An annotation in code for documentation.
  \item \textbf{Array}: A collection of elements, indexed by numbers.
  \item \textbf{String}: A sequence of characters.
  \item \textbf{Integer}: A whole number (e.g., 123)
  \item \textbf{Float}: A number with a decimal point (e.g., 3.14)
  \item \textbf{Boolean}: Represents true or false.
  \item \textbf{Operator}: Performs operations (e.g., +, -, *, /).
  \item \textbf{Variable Assignment}: Set a variable's value.
\end{itemize}

\cheatsheetsection{Glossary}

\textbf{Argument/Parameter}: Information passed to a function or command to influence its behavior.

\textbf{Branch}: A separate line of development within a Git repository, often used for isolating features or bug fixes.

\textbf{Clone}: Creating a local copy of a remote repository on your computer.

\textbf{Command}: An instruction to the shell.

\textbf{Commit}: A snapshot of changes made to the code, accompanied by a message describing the changes.

\textbf{Command Line Interface (CLI)}: A text-based interface for interacting with a computer's operating system and software.

\textbf{Directory}: A folder that can contain files or other directories.

\textbf{Environment Variable}: A variable that holds system or user-specific information used by applications and the operating system.

\textbf{Executable}: A file that can be run as a program or script.

\textbf{File}: A named collection of data or information.

\textbf{Fork}: Creating a personal copy of a repository, allowing independent development without affecting the original.

\textbf{Function}: A reusable block of code that performs a specific task.

\textbf{Git}: A distributed version control system used for tracking changes in code.

\textbf{IDE (Integrated Development Environment)}: A software suite that combines code editing, debugging, and build automation tools for software development.

\textbf{Loop}: A programming construct that repeatedly executes a block of code until a specified condition is met.

\textbf{Merge}: Combining changes from one branch into another.

\textbf{Operator}: A symbol or keyword that performs an operation on one or more values or variables in code.

\textbf{Path}: The address or location of a file or directory in the file system.

\textbf{Permission}: Access rights that determine who can read, write, or execute files and directories.

\textbf{Pipe}: A method for redirecting the output of one command as input to another command.

\textbf{Programming Languages}: Languages used to write software, each with its own syntax and purpose.

\textbf{Pull}: Fetching and incorporating code changes from a remote repository into the local repository.

\textbf{Pull Request (PR)}: A request to merge changes from one branch or fork of a repository into another, often used for code review.

\textbf{Push}: Sending local code changes to a remote repository.

\textbf{Repository}: A storage location for a project's files and version history.

\textbf{Remote}: A copy of a Git repository hosted on a server or another location.

\textbf{Script}: A series of commands or instructions saved in a file for automated execution.

\textbf{Standard Input/Output (stdin/stdout)}: Channels for input (keyboard) and output (screen) of commands in the command line.

\textbf{Syntax}: The set of rules that dictate how code must be structured in a programming language.

\textbf{Text Editor}: A software application for creating and editing text files, often used for code editing.

\textbf{User Directory (Home Directory)}: The top-level directory associated with a user's account, often represented by `~`.

\textbf{Variable}: A symbolic name for a value or data storage location.

\textbf{Version Control}: The practice of tracking and managing changes to code using tools like Git.

\vfill

\cheatsheetfooter{Your Name}{Your GitHub Link}

\end{multicols*}

\end{document}
